%%%%%%%% ICML 2023 EXAMPLE LATEX SUBMISSION FILE %%%%%%%%%%%%%%%%%

\documentclass{article}

% Recommended, but optional, packages for figures and better typesetting:
\usepackage{microtype}
\usepackage{graphicx}
\usepackage{subfigure}
\usepackage{booktabs} % for professional tables

\usepackage{tikz}
% Corporate Design of the University of Tübingen
% Primary Colors
\definecolor{TUred}{RGB}{165,30,55}
\definecolor{TUgold}{RGB}{180,160,105}
\definecolor{TUdark}{RGB}{50,65,75}
\definecolor{TUgray}{RGB}{175,179,183}

% Secondary Colors
\definecolor{TUdarkblue}{RGB}{65,90,140}
\definecolor{TUblue}{RGB}{0,105,170}
\definecolor{TUlightblue}{RGB}{80,170,200}
\definecolor{TUlightgreen}{RGB}{130,185,160}
\definecolor{TUgreen}{RGB}{125,165,75}
\definecolor{TUdarkgreen}{RGB}{50,110,30}
\definecolor{TUocre}{RGB}{200,80,60}
\definecolor{TUviolet}{RGB}{175,110,150}
\definecolor{TUmauve}{RGB}{180,160,150}
\definecolor{TUbeige}{RGB}{215,180,105}
\definecolor{TUorange}{RGB}{210,150,0}
\definecolor{TUbrown}{RGB}{145,105,70}

% hyperref makes hyperlinks in the resulting PDF.
% If your build breaks (sometimes temporarily if a hyperlink spans a page)
% please comment out the following usepackage line and replace
% \usepackage{icml2023} with \usepackage[nohyperref]{icml2023} above.
\usepackage{hyperref}


% Attempt to make hyperref and algorithmic work together better:
\newcommand{\theHalgorithm}{\arabic{algorithm}}

\usepackage[accepted]{icml2023}

% For theorems and such
\usepackage{amsmath}
\usepackage{amssymb}
\usepackage{mathtools}
\usepackage{amsthm}

% if you use cleveref..
\usepackage[capitalize,noabbrev]{cleveref}

%%%%%%%%%%%%%%%%%%%%%%%%%%%%%%%%
% THEOREMS
%%%%%%%%%%%%%%%%%%%%%%%%%%%%%%%%
\theoremstyle{plain}
\newtheorem{theorem}{Theorem}[section]
\newtheorem{proposition}[theorem]{Proposition}
\newtheorem{lemma}[theorem]{Lemma}
\newtheorem{corollary}[theorem]{Corollary}
\theoremstyle{definition}
\newtheorem{definition}[theorem]{Definition}
\newtheorem{assumption}[theorem]{Assumption}
\theoremstyle{remark}
\newtheorem{remark}[theorem]{Remark}

% Todonotes is useful during development; simply uncomment the next line
%    and comment out the line below the next line to turn off comments
%\usepackage[disable,textsize=tiny]{todonotes}
\usepackage[textsize=tiny]{todonotes}


% The \icmltitle you define below is probably too long as a header.
% Therefore, a short form for the running title is supplied here:
\icmltitlerunning{Project Report Template for Data Literacy 2023/24}

\begin{document}

\twocolumn[
\icmltitle{How many Fatal Car Accidents does Preventable Human Error Cause?\\ (TODO: Refactor Title)}
% something along the lines of "what role does preventable human error play?"


% It is OKAY to include author information, even for blind
% submissions: the style file will automatically remove it for you
% unless you've provided the [accepted] option to the icml2023
% package.

% List of affiliations: The first argument should be a (short)
% identifier you will use later to specify author affiliations
% Academic affiliations should list Department, University, City, Region, Country
% Industry affiliations should list Company, City, Region, Country

% You can specify symbols, otherwise they are numbered in order.
% Ideally, you should not use this facility. Affiliations will be numbered
% in order of appearance and this is the preferred way.
\icmlsetsymbol{equal}{*}

\begin{icmlauthorlist}
\icmlauthor{Leon Trochelmann}{equal,first}
\icmlauthor{Jonathan Ranck}{equal,second}
\icmlauthor{Paul-Henrik Heilmann}{equal,third}
\icmlauthor{Filippo Albani}{equal,fourth}
\end{icmlauthorlist}

% fill in your matrikelnummer, email address, degree, for each group member
\icmlaffiliation{first}{Matrikelnummer 6646000, leon.trochelmann@student.uni-tuebingen.de, MSc Machine Learning}
\icmlaffiliation{second}{Matrikelnummer 6230070, jonathan.ranck@student.uni-tuebingen.de, BSc Physics}
\icmlaffiliation{third}{Matrikelnummer 16648314, paul-henrik.heilmann@student.uni-tuebingen.de, MSc Machine Learning}
\icmlaffiliation{fourth}{Matrikelnummer 6638113, filippo.albani@student.uni-tuebingen.de, MSc Physics}

% You may provide any keywords that you
% find helpful for describing your paper; these are used to populate
% the "keywords" metadata in the PDF but will not be shown in the document
\icmlkeywords{Machine Learning, Transportation, Analysis, Accidents}

\vskip 0.3in
]

% this must go after the closing bracket ] following \twocolumn[ ...

% This command actually creates the footnote in the first column
% listing the affiliations and the copyright notice.
% The command takes one argument, which is text to display at the start of the footnote.
% The \icmlEqualContribution command is standard text for equal contribution.
% Remove it (just {}) if you do not need this facility.

%\printAffiliationsAndNotice{}  % leave blank if no need to mention equal contribution
\printAffiliationsAndNotice{\icmlEqualContribution} % otherwise use the standard text.


\begin{abstract}
% Put your abstract here. Abstracts typically start with a sentence motivating why the subject is interesting. Then mention the data, methodology or methods you are working with, and describe results. 

Lorem ipsum dolor sit amet, consectetur adipiscing elit. Nullam condimentum, sapien euismod pulvinar pulvinar, purus erat congue erat, quis auctor orci nisl id quam. Nullam eleifend, tellus at lobortis dignissim, risus metus maximus massa, eget mollis lorem arcu vitae nunc. Donec at sodales elit. Nullam tincidunt sem vitae tellus eleifend condimentum. Ut gravida urna sit amet arcu posuere accumsan. Nullam volutpat dictum nunc, non commodo magna ornare nec. Donec lacinia urna vel libero fermentum, sit amet facilisis enim maximus. Donec bibendum vitae nibh sit amet viverra.
\end{abstract}


\section{Introduction}\label{sec:intro}

% Motivate the problem, situation or topic you decided to work on. Describe why it matters (is it of societal, economic, scientific value?). Outline the rest of the paper (use references, e.g.~to \Cref{sec:methods}: What kind of data you are working with, how you analyse it, and what kind of conclusion you reached. The point of the introduction is to make the reader want to read the rest of the paper.

Suspendisse potenti. Curabitur dictum eget risus ut dignissim. Cras placerat tellus a aliquet sagittis. Aliquam aliquet, dui sed venenatis ultricies, nunc mauris tempus justo, a suscipit turpis turpis et nulla. Curabitur tincidunt hendrerit elementum. Suspendisse eget justo rutrum ligula suscipit scelerisque. Donec eget tincidunt enim, vitae ullamcorper dolor. Quisque consequat neque in lorem finibus iaculis. Integer sit amet leo ipsum. Sed tristique malesuada rutrum. Vestibulum purus erat, rhoncus eu erat eu, condimentum consectetur sapien. Proin id aliquet est.

% Section Introduction:
% - (status quo) Mention that self-driving cars are often motivated by an assumption that car accidents are avoidable and caused by "human error"
% - (problem with the status quo) However, human error is never clearly defined and claims that "90% of car accidents are caused by human error" have no basis to substantiate them
% - (proposed improvment) This is where our project comes in, offering a way to quantify certain preventable human error of which we can be sure that it would not be made by automated cars
% - Outline the paper
%     - Mention which data was used and to what end
%     - Mention which methods were used and to what end
% - Very briefly summarise the takeaways from the results and their implication


\section{Data and Methods}\label{sec:methods}

% In this section, describe \emph{what you did}. Roughly speaking, explain what data you worked with, how or from where it was collected, it's structure and size. Explain your analysis, and any specific choices you made in it. Depending on the nature of your project, you may focus more or less on certain aspects. If you collected data yourself, explain the collection process in detail. If you downloaded data from the net, show an exploratory analysis that builds intuition for the data, and shows that you know the data well. If you are doing a custom analysis, explain how it works and why it is the right choice. If you are using a standard tool, it may still help to briefly outline it. Cite relevant works. You can use the \verb|\citep| and \verb|\citet| commands for this purpose \citep{mackay2003information}.

% This is the template for a figure from the original ICML submission pack. In lecture 10 we will discuss plotting in detail.
% Refer to this lecture on how to include figures in this text.
% 
% \begin{figure}[ht]
% \vskip 0.2in
% \begin{center}
% \centerline{\includegraphics[width=\columnwidth]{icml_numpapers}}
% \caption{Historical locations and number of accepted papers for International
% Machine Learning Conferences (ICML 1993 -- ICML 2008) and International
% Workshops on Machine Learning (ML 1988 -- ML 1992). At the time this figure was
% produced, the number of accepted papers for ICML 2008 was unknown and instead
% estimated.}
% \label{icml-historical}
% \end{center}
% \vskip -0.2in
% \end{figure}

Donec venenatis nisi libero, et ornare quam eleifend rhoncus. Maecenas gravida tincidunt dolor, in congue dolor ornare in. Suspendisse orci risus, semper vitae egestas vel, auctor ornare neque. Aenean cursus sed dui a elementum. Vestibulum in porttitor sem. Sed lacus ligula, convallis non odio in, ultrices fringilla felis. Duis elementum sodales mi a varius. Nunc feugiat, augue ac commodo vulputate, est nunc varius massa, quis fringilla risus tellus sit amet neque. Vestibulum posuere mauris rhoncus tortor sodales semper. Maecenas quam ipsum, dignissim non neque et, luctus commodo libero. Mauris sed ante facilisis ipsum fringilla scelerisque. Cras imperdiet nisl non neque faucibus consequat. Duis feugiat tortor a libero dapibus accumsan. Maecenas pharetra nibh et justo porta, in ultrices enim egestas. 

% Section Data and Methods:
% 	- Outline the section and mention the core idea of our approach
% 	Subsection Data:
%		- Introduce the data properly and in more detail (replacement for missing background section)
%		- Elaborate on aspects of the data relevant to us and tell why they're relevant
% 	Subsection Methods:
%		- Maybe make subsubsections for 1. human error definition and 2. analysis
%		- Describe our definition of human error in detail
%		- Describe our analysis on the role of human error as analysis of conditional probabilities
%		- Mention the two phases of our analysis in terms of first finding large human error correlation, and then further analysing which specific errors contribute to the high correlation
%		- Describe the actual experiments  one by one


\section{Results}\label{sec:results}

% In this section outline your results. At this point, you are just stating the outcome of your analysis. You can highlight important aspects (``we observe a significantly higher value of $x$ over $y$''), but leave interpretation and opinion to the next section. This section absoultely \emph{has} to include at least two figures.

Mauris pharetra a urna at faucibus. Donec eu justo magna. Sed viverra dolor at sapien congue, nec tempus justo aliquet. Ut et nisi id odio aliquam iaculis. Sed sit amet eros nisi. Integer dictum risus ac orci sagittis, ut dapibus mauris semper. Pellentesque ac risus quam. Morbi et orci sed mi maximus aliquet sed sit amet leo. 

% Section Results:
%	- In the same order and structure as they were outlined in Section Data and Methods, present the results
%	- Do not draw any conclusions, simply present and document figures and highlight their notable aspects


\section{Discussion \& Conclusion}\label{sec:conclusion}

% Use this section to briefly summarize the entire text. Highlight limitations and problems, but also make clear statements where they are possible and supported by the analysis. 

Suspendisse sed arcu malesuada, tempor libero ut, hendrerit nisl. Aliquam quis elit dictum, hendrerit leo sed, tempor eros. Donec rutrum eu diam ac malesuada. Nullam non pretium velit. Sed hendrerit, magna vitae vehicula volutpat, magna risus commodo ligula, interdum iaculis metus tellus at enim. Suspendisse laoreet euismod nulla a laoreet.

% Section Discussion and Conclusion:
%	- Subsection Discussion:
%		- In the order and structure in which the experiments and results were presented, go through their implications and meaning in context
%		- At the very end, give a summarising statement combining all of the discussion into a key takeaway (or a few)
%	- Subsection Conclusion:
%		- Give a very concise summary of the paper
%		- If applicable, discuss possible future work and perspectives


\section*{Contribution Statement}

% Explain here, in one sentence per person, what each group member contributed. For example, you could write: Max Mustermann collected and prepared data. Gabi Musterfrau and John Doe performed the data analysis. Jane Doe produced visualizations. All authors will jointly wrote the text of the report. Note that you, as a group, a collectively responsible for the report. Your contributions should be roughly equal in amount and difficulty.
Etiam aliquet hendrerit lectus porttitor placerat. Nulla ac nisl enim. Nunc vulputate augue vitae erat hendrerit tempus. Proin dolor nulla, accumsan at risus ac, volutpat bibendum nulla. Donec porttitor varius facilisis. Praesent at urna nulla. Duis eleifend nulla id quam auctor mollis. Nulla facilisi.\\
\\
Citation to not break bibtex: \citep{mackay2003information}


% \section*{Notes} 

% Your entire report has a \textbf{hard page limit of 4 pages} excluding references. (I.e. any pages beyond page 4 must only contain references). Appendices are \emph{not} possible. But you can put additional material, like interactive visualizations or videos, on a githunb repo (use \href{https://github.com/pnkraemer/tueplots}{links} in your pdf to refer to them). Each report has to contain \textbf{at least three plots or visualizations}, and \textbf{cite at least two references}. More details about how to prepare the report, inclucing how to produce plots, cite correctly, and how to ideally structure your github repo, will be discussed in the lecture, where a rubric for the evaluation will also be provided.

\bibliography{bibliography}
\bibliographystyle{icml2023}

\end{document}


% This document was modified from the file originally made available by
% Pat Langley and Andrea Danyluk for ICML-2K. This version was created
% by Iain Murray in 2018, and modified by Alexandre Bouchard in
% 2019 and 2021 and by Csaba Szepesvari, Gang Niu and Sivan Sabato in 2022.
% Modified again in 2023 by Sivan Sabato and Jonathan Scarlett.
% Previous contributors include Dan Roy, Lise Getoor and Tobias
% Scheffer, which was slightly modified from the 2010 version by
% Thorsten Joachims & Johannes Fuernkranz, slightly modified from the
% 2009 version by Kiri Wagstaff and Sam Roweis's 2008 version, which is
% slightly modified from Prasad Tadepalli's 2007 version which is a
% lightly changed version of the previous year's version by Andrew
% Moore, which was in turn edited from those of Kristian Kersting and
% Codrina Lauth. Alex Smola contributed to the algorithmic style files.
